\section{Ti\-Xml\-Handle Class Reference}
\label{classTiXmlHandle}\index{TiXmlHandle@{TiXmlHandle}}
A Ti\-Xml\-Handle is a class that wraps a node pointer with null checks; this is an incredibly useful thing.  


{\tt \#include $<$tinyxml.h$>$}

\subsection*{Public Member Functions}
\begin{CompactItemize}
\item 
{\bf Ti\-Xml\-Handle} ({\bf Ti\-Xml\-Node} $\ast$\_\-node)\label{classTiXmlHandle_TiXmlHandlea0}

\begin{CompactList}\small\item\em Create a handle from any node (at any depth of the tree.) This can be a null pointer. \item\end{CompactList}\item 
{\bf Ti\-Xml\-Handle} (const {\bf Ti\-Xml\-Handle} \&ref)\label{classTiXmlHandle_TiXmlHandlea1}

\begin{CompactList}\small\item\em Copy constructor. \item\end{CompactList}\item 
{\bf Ti\-Xml\-Handle} {\bf operator=} (const {\bf Ti\-Xml\-Handle} \&ref)\label{classTiXmlHandle_TiXmlHandlea2}

\item 
{\bf Ti\-Xml\-Handle} {\bf First\-Child} () const\label{classTiXmlHandle_TiXmlHandlea3}

\begin{CompactList}\small\item\em Return a handle to the first child node. \item\end{CompactList}\item 
{\bf Ti\-Xml\-Handle} {\bf First\-Child} (const char $\ast$value) const\label{classTiXmlHandle_TiXmlHandlea4}

\begin{CompactList}\small\item\em Return a handle to the first child node with the given name. \item\end{CompactList}\item 
{\bf Ti\-Xml\-Handle} {\bf First\-Child\-Element} () const\label{classTiXmlHandle_TiXmlHandlea5}

\begin{CompactList}\small\item\em Return a handle to the first child element. \item\end{CompactList}\item 
{\bf Ti\-Xml\-Handle} {\bf First\-Child\-Element} (const char $\ast$value) const\label{classTiXmlHandle_TiXmlHandlea6}

\begin{CompactList}\small\item\em Return a handle to the first child element with the given name. \item\end{CompactList}\item 
{\bf Ti\-Xml\-Handle} {\bf Child} (const char $\ast$value, int index) const
\begin{CompactList}\small\item\em Return a handle to the \char`\"{}index\char`\"{} child with the given name. \item\end{CompactList}\item 
{\bf Ti\-Xml\-Handle} {\bf Child} (int index) const
\begin{CompactList}\small\item\em Return a handle to the \char`\"{}index\char`\"{} child. \item\end{CompactList}\item 
{\bf Ti\-Xml\-Handle} {\bf Child\-Element} (const char $\ast$value, int index) const
\begin{CompactList}\small\item\em Return a handle to the \char`\"{}index\char`\"{} child element with the given name. \item\end{CompactList}\item 
{\bf Ti\-Xml\-Handle} {\bf Child\-Element} (int index) const
\begin{CompactList}\small\item\em Return a handle to the \char`\"{}index\char`\"{} child element. \item\end{CompactList}\item 
{\bf Ti\-Xml\-Handle} {\bf First\-Child} (const std::string \&\_\-value) const\label{classTiXmlHandle_TiXmlHandlea11}

\item 
{\bf Ti\-Xml\-Handle} {\bf First\-Child\-Element} (const std::string \&\_\-value) const\label{classTiXmlHandle_TiXmlHandlea12}

\item 
{\bf Ti\-Xml\-Handle} {\bf Child} (const std::string \&\_\-value, int index) const\label{classTiXmlHandle_TiXmlHandlea13}

\item 
{\bf Ti\-Xml\-Handle} {\bf Child\-Element} (const std::string \&\_\-value, int index) const\label{classTiXmlHandle_TiXmlHandlea14}

\item 
{\bf Ti\-Xml\-Node} $\ast$ {\bf Node} () const\label{classTiXmlHandle_TiXmlHandlea15}

\begin{CompactList}\small\item\em Return the handle as a {\bf Ti\-Xml\-Node}{\rm (p.\,\pageref{classTiXmlNode})}. This may return null. \item\end{CompactList}\item 
{\bf Ti\-Xml\-Element} $\ast$ {\bf Element} () const\label{classTiXmlHandle_TiXmlHandlea16}

\begin{CompactList}\small\item\em Return the handle as a {\bf Ti\-Xml\-Element}{\rm (p.\,\pageref{classTiXmlElement})}. This may return null. \item\end{CompactList}\item 
{\bf Ti\-Xml\-Text} $\ast$ {\bf Text} () const\label{classTiXmlHandle_TiXmlHandlea17}

\begin{CompactList}\small\item\em Return the handle as a {\bf Ti\-Xml\-Text}{\rm (p.\,\pageref{classTiXmlText})}. This may return null. \item\end{CompactList}\item 
{\bf Ti\-Xml\-Unknown} $\ast$ {\bf Unknown} () const\label{classTiXmlHandle_TiXmlHandlea18}

\begin{CompactList}\small\item\em Return the handle as a {\bf Ti\-Xml\-Unknown}{\rm (p.\,\pageref{classTiXmlUnknown})}. This may return null;. \item\end{CompactList}\end{CompactItemize}


\subsection{Detailed Description}
A Ti\-Xml\-Handle is a class that wraps a node pointer with null checks; this is an incredibly useful thing. 

Note that Ti\-Xml\-Handle is not part of the Tiny\-Xml DOM structure. It is a separate utility class.

Take an example: 

\footnotesize\begin{verbatim}
	<Document>
		<Element attributeA = "valueA">
			<Child attributeB = "value1" />
			<Child attributeB = "value2" />
		</Element>
	<Document>
	\end{verbatim}
\normalsize


Assuming you want the value of \char`\"{}attribute\-B\char`\"{} in the 2nd \char`\"{}Child\char`\"{} element, it's very easy to write a $\ast$lot$\ast$ of code that looks like:



\footnotesize\begin{verbatim}
	TiXmlElement* root = document.FirstChildElement( "Document" );
	if ( root )
	{
		TiXmlElement* element = root->FirstChildElement( "Element" );
		if ( element )
		{
			TiXmlElement* child = element->FirstChildElement( "Child" );
			if ( child )
			{
				TiXmlElement* child2 = child->NextSiblingElement( "Child" );
				if ( child2 )
				{
					// Finally do something useful.
	\end{verbatim}
\normalsize


And that doesn't even cover \char`\"{}else\char`\"{} cases. Ti\-Xml\-Handle addresses the verbosity of such code. A Ti\-Xml\-Handle checks for null pointers so it is perfectly safe and correct to use:



\footnotesize\begin{verbatim}
	TiXmlHandle docHandle( &document );
	TiXmlElement* child2 = docHandle.FirstChild( "Document" ).FirstChild( "Element" ).Child( "Child", 1 ).Element();
	if ( child2 )
	{
		// do something useful
	\end{verbatim}
\normalsize


Which is MUCH more concise and useful.

It is also safe to copy handles - internally they are nothing more than node pointers. 

\footnotesize\begin{verbatim}
	TiXmlHandle handleCopy = handle;
	\end{verbatim}
\normalsize


What they should not be used for is iteration:



\footnotesize\begin{verbatim}
	int i=0; 
	while ( true )
	{
		TiXmlElement* child = docHandle.FirstChild( "Document" ).FirstChild( "Element" ).Child( "Child", i ).Element();
		if ( !child )
			break;
		// do something
		++i;
	}
	\end{verbatim}
\normalsize


It seems reasonable, but it is in fact two embedded while loops. The Child method is a linear walk to find the element, so this code would iterate much more than it needs to. Instead, prefer:



\footnotesize\begin{verbatim}
	TiXmlElement* child = docHandle.FirstChild( "Document" ).FirstChild( "Element" ).FirstChild( "Child" ).Element();

	for( child; child; child=child->NextSiblingElement() )
	{
		// do something
	}
	\end{verbatim}
\normalsize




\subsection{Member Function Documentation}
\index{TiXmlHandle@{Ti\-Xml\-Handle}!Child@{Child}}
\index{Child@{Child}!TiXmlHandle@{Ti\-Xml\-Handle}}
\subsubsection{\setlength{\rightskip}{0pt plus 5cm}{\bf Ti\-Xml\-Handle} Ti\-Xml\-Handle::Child (int {\em index}) const}\label{classTiXmlHandle_TiXmlHandlea8}


Return a handle to the \char`\"{}index\char`\"{} child. 

The first child is 0, the second 1, etc.\index{TiXmlHandle@{Ti\-Xml\-Handle}!Child@{Child}}
\index{Child@{Child}!TiXmlHandle@{Ti\-Xml\-Handle}}
\subsubsection{\setlength{\rightskip}{0pt plus 5cm}{\bf Ti\-Xml\-Handle} Ti\-Xml\-Handle::Child (const char $\ast$ {\em value}, int {\em index}) const}\label{classTiXmlHandle_TiXmlHandlea7}


Return a handle to the \char`\"{}index\char`\"{} child with the given name. 

The first child is 0, the second 1, etc.\index{TiXmlHandle@{Ti\-Xml\-Handle}!ChildElement@{ChildElement}}
\index{ChildElement@{ChildElement}!TiXmlHandle@{Ti\-Xml\-Handle}}
\subsubsection{\setlength{\rightskip}{0pt plus 5cm}{\bf Ti\-Xml\-Handle} Ti\-Xml\-Handle::Child\-Element (int {\em index}) const}\label{classTiXmlHandle_TiXmlHandlea10}


Return a handle to the \char`\"{}index\char`\"{} child element. 

The first child element is 0, the second 1, etc. Note that only Ti\-Xml\-Elements are indexed: other types are not counted.\index{TiXmlHandle@{Ti\-Xml\-Handle}!ChildElement@{ChildElement}}
\index{ChildElement@{ChildElement}!TiXmlHandle@{Ti\-Xml\-Handle}}
\subsubsection{\setlength{\rightskip}{0pt plus 5cm}{\bf Ti\-Xml\-Handle} Ti\-Xml\-Handle::Child\-Element (const char $\ast$ {\em value}, int {\em index}) const}\label{classTiXmlHandle_TiXmlHandlea9}


Return a handle to the \char`\"{}index\char`\"{} child element with the given name. 

The first child element is 0, the second 1, etc. Note that only Ti\-Xml\-Elements are indexed: other types are not counted.

The documentation for this class was generated from the following files:\begin{CompactItemize}
\item 
src/tinyxml.h\item 
src/tinyxml.cpp\end{CompactItemize}
